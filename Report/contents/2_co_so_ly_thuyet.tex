\section{Cơ sở lý thuyết}

\subsection{Công nghệ nhận diện khuôn mặt}
\hspace{0.5cm}Công nghệ nhận diện khuôn mặt là một lĩnh vực của thị giác máy tính và học sâu, bao gồm các bước chính:

\begin{itemize}
    \item \textbf{Phát hiện khuôn mặt:}
    \begin{itemize}
        \item \textit{Haar Cascade:} Phương pháp truyền thống sử dụng bộ lọc Haar và AdaBoost
        \item \textit{MTCNN (Multi-Task Cascaded CNN):} Mạng neural 3 giai đoạn cho phát hiện và căn chỉnh
        \item \textit{RetinaFace:} Mô hình hiện đại với độ chính xác cao trong nhiều điều kiện
        \item Xử lý các thách thức: góc nghiêng, ánh sáng, che khuất
    \end{itemize}
    
    \item \textbf{Căn chỉnh khuôn mặt:}
    \begin{itemize}
        \item \textit{Facial Landmarks:} Xác định 68 điểm đặc trưng trên khuôn mặt
        \item \textit{Affine Transformation:} Chuẩn hóa vị trí và góc nghiêng
        \item \textit{Face Alignment:} Căn chỉnh theo mắt và mũi
        \item Xử lý biến dạng và chuẩn hóa kích thước
    \end{itemize}
    
    \item \textbf{Trích xuất đặc trưng:}
    \begin{itemize}
        \item \textit{Deep CNN Architecture:} Sử dụng mạng ResNet-50 hoặc MobileNetV2
        \item \textit{Feature Embedding:} Vector 512 chiều đặc trưng cho mỗi khuôn mặt
        \item \textit{Triplet Loss:} Huấn luyện để tối ưu khoảng cách giữa các embedding
        \item Tăng cường dữ liệu và xử lý nhiễu
    \end{itemize}
    
    \item \textbf{So khớp khuôn mặt:}
    \begin{itemize}
        \item \textit{Cosine Similarity:} Đo độ tương đồng giữa các vector đặc trưng
        \item \textit{Threshold Optimization:} Cân bằng giữa false positive và false negative
        \item \textit{Face Verification:} Xác thực danh tính dựa trên ngưỡng tương đồng
        \item Xử lý các trường hợp đặc biệt và ngoại lệ
    \end{itemize}
\end{itemize}

\subsection{Kiến trúc IoT và Edge Computing}
\hspace{0.5cm}Hệ thống sử dụng kiến trúc IoT kết hợp với edge computing:

\begin{itemize}
    \item \textbf{Edge Device (ESP32):}
    \begin{itemize}
        \item \textit{Hardware Architecture:} 
        \begin{itemize}
            \item CPU: Xtensa dual-core 32-bit LX6 (240MHz)
            \item RAM: 520KB SRAM
            \item Flash: 4MB external flash
            \item Camera: OV2640 2MP sensor
            \item Display: ST7789 1.3" TFT LCD
        \end{itemize}
        \item \textit{Software Stack:}
        \begin{itemize}
            \item FreeRTOS kernel
            \item ESP-IDF framework
            \item MQTT client library
            \item TFT_eSPI graphics library
        \end{itemize}
        \item \textit{Edge Processing:}
        \begin{itemize}
            \item Image capture và preprocessing
            \item Basic face detection
            \item Data compression và buffering
            \item Local decision making
        \end{itemize}
    \end{itemize}
    
    \item \textbf{Gateway Layer:}
    \begin{itemize}
        \item \textit{Hardware Requirements:}
        \begin{itemize}
            \item Multi-core processor
            \item Minimum 4GB RAM
            \item SSD storage
            \item Stable network connection
        \end{itemize}
        \item \textit{Software Components:}
        \begin{itemize}
            \item MQTT broker (Mosquitto)
            \item Python processing service
            \item Redis cache
            \item Monitoring agents
        \end{itemize}
        \item \textit{Processing Pipeline:}
        \begin{itemize}
            \item Image decompression
            \item Quality assessment
            \item Feature extraction
            \item Load balancing
        \end{itemize}
    \end{itemize}
\end{itemize}

\subsection{Microservices và Container}
\hspace{0.5cm}Kiến trúc microservices được áp dụng để tăng tính module hóa và khả năng mở rộng:

\begin{itemize}
    \item \textbf{Service Architecture:}
    \begin{itemize}
        \item \textit{Core Services:}
        \begin{itemize}
            \item Authentication Service
            \item Face Recognition Service
            \item Device Management Service
            \item Data Analytics Service
        \end{itemize}
        \item \textit{Supporting Services:}
        \begin{itemize}
            \item Logging Service
            \item Monitoring Service
            \item Configuration Service
            \item Notification Service
        \end{itemize}
    \end{itemize}
    
    \item \textbf{Container Management:}
    \begin{itemize}
        \item \textit{Docker Configuration:}
        \begin{itemize}
            \item Multi-stage builds
            \item Layer optimization
            \item Security hardening
            \item Resource limits
        \end{itemize}
        \item \textit{Orchestration:}
        \begin{itemize}
            \item Service discovery
            \item Load balancing
            \item Health monitoring
            \item Auto-scaling
        \end{itemize}
    \end{itemize}
\end{itemize}

\subsection{Bảo mật và Xác thực}
\hspace{0.5cm}Hệ thống tích hợp nhiều lớp bảo mật:

\begin{itemize}
    \item \textbf{Authentication System:}
    \begin{itemize}
        \item \textit{JWT Implementation:}
        \begin{itemize}
            \item Token generation và validation
            \item Refresh token mechanism
            \item Token revocation
            \item Session management
        \end{itemize}
        \item \textit{Access Control:}
        \begin{itemize}
            \item Role-based permissions
            \item Resource-level access
            \item API authorization
            \item Audit logging
        \end{itemize}
    \end{itemize}
    
    \item \textbf{Data Security:}
    \begin{itemize}
        \item \textit{Encryption:}
        \begin{itemize}
            \item AES-256 for data at rest
            \item TLS 1.3 for transmission
            \item Key management
            \item Secure storage
        \end{itemize}
        \item \textit{Privacy Protection:}
        \begin{itemize}
            \item Data anonymization
            \item GDPR compliance
            \item Data retention policies
            \item Consent management
        \end{itemize}
    \end{itemize}
\end{itemize}

\subsection{Công nghệ Web hiện đại}
\hspace{0.5cm}Frontend được phát triển với các công nghệ web tiên tiến:

\begin{itemize}
    \item \textbf{Angular Framework:}
    \begin{itemize}
        \item \textit{Architecture:}
        \begin{itemize}
            \item Modular design
            \item Component hierarchy
            \item Service injection
            \item Routing system
        \end{itemize}
        \item \textit{State Management:}
        \begin{itemize}
            \item NgRx store
            \item Effects handling
            \item Action creators
            \item Selectors
        \end{itemize}
    \end{itemize}
    
    \item \textbf{Real-time Features:}
    \begin{itemize}
        \item \textit{WebSocket Implementation:}
        \begin{itemize}
            \item Socket.io integration
            \item Event handling
            \item Reconnection logic
            \item Message queuing
        \end{itemize}
        \item \textit{Push Notifications:}
        \begin{itemize}
            \item Service Workers
            \item Push API
            \item Notification API
            \item Background sync
        \end{itemize}
    \end{itemize}
    
    \item \textbf{Progressive Enhancement:}
    \begin{itemize}
        \item \textit{PWA Features:}
        \begin{itemize}
            \item Offline support
            \item App manifest
            \item Cache strategies
            \item Installation flow
        \end{itemize}
        \item \textit{Performance Optimization:}
        \begin{itemize}
            \item Lazy loading
            \item Code splitting
            \item Tree shaking
            \item Bundle optimization
        \end{itemize}
    \end{itemize}
\end{itemize} 