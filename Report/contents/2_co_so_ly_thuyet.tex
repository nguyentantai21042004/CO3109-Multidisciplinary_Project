\section{Cơ sở lý thuyết}

\subsection{Công nghệ nhận diện khuôn mặt}
\hspace{0.5cm}Công nghệ nhận diện khuôn mặt là một lĩnh vực của thị giác máy tính và học sâu, bao gồm các bước chính:

\begin{itemize}
    \item \textbf{Phát hiện khuôn mặt:} Sử dụng các thuật toán như Haar Cascade hoặc MTCNN
    \item \textbf{Căn chỉnh khuôn mặt:} Xác định và chuẩn hóa vị trí các điểm đặc trưng
    \item \textbf{Trích xuất đặc trưng:} Sử dụng mạng neural tích chập (CNN) để tạo vector đặc trưng
    \item \textbf{So khớp khuôn mặt:} Tính toán độ tương đồng giữa các vector đặc trưng
\end{itemize}

\subsection{Kiến trúc IoT và Edge Computing}
\hspace{0.5cm}Hệ thống sử dụng kiến trúc IoT kết hợp với edge computing:

\begin{itemize}
    \item \textbf{Edge Device (ESP32):}
    \begin{itemize}
        \item Vi xử lý lõi kép
        \item Tích hợp Wi-Fi và Bluetooth
        \item Xử lý ảnh cơ bản tại thiết bị
        \item Giao tiếp với gateway qua MQTT
    \end{itemize}
    
    \item \textbf{Gateway Layer:}
    \begin{itemize}
        \item Xử lý ảnh trung gian
        \item Điều phối luồng dữ liệu
        \item Cache và buffer tạm thời
        \item Giao tiếp hai chiều với thiết bị
    \end{itemize}
\end{itemize}

\subsection{Microservices và Container}
\hspace{0.5cm}Kiến trúc microservices được áp dụng để tăng tính module hóa và khả năng mở rộng:

\begin{itemize}
    \item \textbf{Container hóa:}
    \begin{itemize}
        \item Sử dụng Docker để đóng gói dịch vụ
        \item Quản lý với Docker Compose
        \item Cô lập môi trường triển khai
        \item Dễ dàng scale và update
    \end{itemize}
    
    \item \textbf{Service Discovery:}
    \begin{itemize}
        \item Đăng ký và khám phá dịch vụ
        \item Load balancing
        \item Health checking
        \item Fault tolerance
    \end{itemize}
\end{itemize}

\subsection{Bảo mật và Xác thực}
\hspace{0.5cm}Hệ thống tích hợp nhiều lớp bảo mật:

\begin{itemize}
    \item \textbf{JWT Authentication:}
    \begin{itemize}
        \item Token-based authentication
        \item Refresh token mechanism
        \item Role-based access control
    \end{itemize}
    
    \item \textbf{Bảo mật dữ liệu:}
    \begin{itemize}
        \item Mã hóa dữ liệu nhạy cảm
        \item HTTPS cho API endpoints
        \item Secure WebSocket cho real-time
    \end{itemize}
    
    \item \textbf{Device Security:}
    \begin{itemize}
        \item Xác thực thiết bị với API key
        \item Mã hóa kênh MQTT
        \item Giới hạn quyền truy cập
    \end{itemize}
\end{itemize}

\subsection{Công nghệ Web hiện đại}
\hspace{0.5cm}Frontend được phát triển với các công nghệ web tiên tiến:

\begin{itemize}
    \item \textbf{Angular Framework:}
    \begin{itemize}
        \item Component-based architecture
        \item Reactive programming với RxJS
        \item State management
        \item Lazy loading modules
    \end{itemize}
    
    \item \textbf{Real-time Features:}
    \begin{itemize}
        \item WebSocket cho cập nhật real-time
        \item Server-Sent Events
        \item Push notifications
    \end{itemize}
    
    \item \textbf{UI/UX Design:}
    \begin{itemize}
        \item Material Design
        \item Responsive layout
        \item Progressive Web App (PWA)
    \end{itemize}
\end{itemize} 